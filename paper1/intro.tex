\section{introduction}

Active galactic nucleii are thought to be powered through the
accretion of matter from the proximal environment into a supermassive
black hole.  The radiation emitted excites the surrounding medium
which becomes detectable as narrow line regions, broad line regions
and optical continuum.  Moreover, in the direction perpendicular to
the accretion disk, where the medium is more transparent, jets will
appear.  If a jet is oriented towards the Earth, a quasar is observed
\citep[e.g.,][]{1984RvMP...56..255B}.  While the basic mechanism
\citep[originating in the work
  of][]{1964ApJ...140..796S,1964SPhD....9..246Z,1969Natur.223..690L}
is not in doubt, the central engines, near the event horizons of the
black holes, remain to be probed.

For the black holes in the Galactic centre and the centre of M87
---the two nearest objects that barely qualify as AGN--- the central
engines (which are $<0.1\rm\,mas$ on the sky) are close to being
resolved through very long baseline interferometry, which shows
preliminary indications of the jet-launching structures
\citep{2008JPhCS.131a2055D,2012Sci...338..355D,2013MNRAS.434..765K,2016arXiv160205527F}.
Current data do not deliver images, but require fitting to to
predefined models for the images.  A whole range of models have been
applied, starting from simple geometric models to more complex
physical models
\citep{2008Natur.455...78D,2011ApJ...738...38B,2009ApJ...706..497M,2010ApJ...717.1092D}.
The more complicated models nonetheless tend to predict a crescent
shaped silhouette of the black hole.  This motivated
\cite{2013MNRAS.434..765K} to use a simple geometric crescent model to
fit the data, and argued that the crescent is nothing but the
silhouette of the event horizon.

The great majority of quasars, however, lie at redshifts beyond 2
\citep{2014A&A...563A..54P} and their central engines would be orders
of magnitude smaller on the sky. The direct observations of the black
hole silhouettes of quasars is far beyond foreseeable instrumentation.
In the present paper we consider a possible indirect method, related
to \cite{1999ApJ...524...49A}, through which gravitational
microlensing could probe the black hole shadow and its proximal quasar
environment.

Gravitational microlensing of quasars, reviewed in
Section~\ref{sec:microlensing} below, refers to sharp changes in the
observed brightness of quasars that have been lensed by intervening
galaxies, without any changes in the intrinsic luminosity.
Microlensing affects only the light from the innermost part of the
quasar, such as the optical continuum
\citep[e.g.,][]{2012A&A...544A..62S} and is a consequence of two
things: the very small size of the central engine, and granularity of
the mass distribution of a lensing galaxy due to stars.  The latter
means that the brightness of a lensed object (or lensing
magnification) is not a smooth function of source position, but a
contains a complicated network of singular curves, known as caustics.
Figure~\ref{fig:magnification_map} shows part of a magnification map
with a few caustics.  The lensed brightness would be given by placing
the source on such a magnification map and integrating the surface
brightness weighted by the magnification.  Most astrophysical sources
straddle several caustics, and hence their net brightness varies
smoothly with location.  The central engine of quasars, however, is
smaller than the typical spacing between caustics.  As a result, the
lensed brightness undergoes sudden changes as a quasar crosses a
caustic.  The effect supplies an upper limit on the size of the
central engine, and can also be used to study the mass distribution
and kinematics of stars in the lensing galaxy as well
\citep[e.g.,][]{2012ApJ...744..111P}.  Caustics have an additional
remarkable feature: though they can be very complicated, they have
some universal properties well-known from catastrophe theory.  In
particular, very close to the simplest caustics (known as folds), the
magnification is approximately constant on one side and
$\propto1/\sqrt p$ where $p$ is the transverse distance of the source
from the caustic.  This property will be exploited later.

In Section~\ref{sec:source-models} we introduce the three source
profiles used in our subsequent models and simulations: a
constant-brightness disc, a circular Gaussian, and the crescent source
introduced by \citep{2013MNRAS.434..765K}.  The latter is simply a
constant-brightness disc with a smaller, non-concentric disc cut out
of it.  We also derive the half-light radius for a crescent.  The
half-light radius can characterise the source size for all three types
of source.

Section~\ref{sec:fold-crossing} shows the light curves that result
when each of the model sources crosses an ideal fold.  This would
apply in Figure~\ref{fig:magnification_map} to sources along the path
AB or BC, for sources small enough that the curvature of the caustics
is negligible.  With this assumption one can imagine the caustic as an
infinite wall to be crossed by the source as presented in
Figure~\ref{fig:infinite_fold}.  The source brightness distribution
parallel to the caustic naturally makes no difference to the
observable brightness; each source can be replaced by an effective
one-dimensional source profile, by flattening the source so it becomes
perpendicular to the caustic.  In principle, the effective
one-dimensional brightness profile could be recovered from the light
curve by deconvolution.  \cite{1999ApJ...524...49A} modelled this
profile as the result of a circular accretion disc seen through the
spacetime around a Kerr black hole, and \cite{2012MNRAS.423..676A}
have applied the idea to observed light curves to infer properties of
quasar accretion discs.  In this work we take a simpler but arguably
more robust approach: we study features in the light curves
characteristic of a crescent-like source which in turn would indicate
a black-hole silhouette.  Figure~\ref{fig:char_points} shows the
qualitative features: there is a period during which the dark cutout
disc is crossing the caustic, and before and after there are periods
when the only the bright parts of the crescent are in transit across
the caustic.  The details depend on the orientation of the crescent,
but basically the dark disc causes a rising light curve to plateau or
dip.

The crescent characteristics identified in
Figure~\ref{fig:char_points} are from an idealised source crossing an
idealised caustic.  In Section~\ref{sec:numerics} we see what happens
when we replace both of these with numerical simulations.  The ideal
fold-transit is replaced by the path AB in
Figure~\ref{fig:magnification_map}.  The simple crescent is replaced
by a source (Figure~\ref{fig:M87_image}) from published simulations of
the black-hole environment of M87
\citep{2009MNRAS.394L.126M,2012MNRAS.421.1517D}.  The simulated light
curve so obtained (lower panel of Figure~\ref{fig:M87_plots}) still
shows, albeit faintly, features from the crescent.

In Section~\ref{sec:fitting} we carry out source fitting to
lightcurves, with both noise and systematic errors are present.  We
generate three lightcurves by taking a uniform disc, a circular
Gaussian, and a crescent across the path AB in
Figure~\ref{fig:magnification_map}, and then adding noise.  The path
simulates crossing a clean but not ideal fold.  In addition, we
generate templates lightcurves by running the three source types, with
various parameter values, across the path CB.  That is, the templates
come from a similar but not identical caustic, thus deliberately
generating a systematic error.  We then fit the noisy lightcurves to
the templates using Markov chain Monte-Carlo.  We find that the
correct source type can be inferred from the $\chi^2$ values.  The
parameter values can also be inferred. The fitting errors are larger
than the formal uncertainties, which is expected in the presence of
systematic errors, but still appear acceptable.


