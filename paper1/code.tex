\section{Microlensing code and magnification maps}
For the numerical computations of lightcurves for the models a and b, the microlensing code by Joachim Wambsgans was used, which is described in Wambsganss 1999. This code is based on the hierachical tree technique to calculate the gravitational effect of the lensing masses(--> references). The underlying idea is that a microlensing situation can be separated into three planes, the observer, the lens plane and the source plane (assuming the thin lens approximation to be valid). The actual effect, namely the distortion of lightrays emitted from a distant sources, on the way to the observer, is numerically computed by treating this situation from the opposite side: lightrays are shot backward from the observer to lens plane, where they become deflected according to the angle, which is given by the 2-dim. mass distribution in the lens plane.
\begin{equation}
\vec{\alpha}_{i}=\frac{4G}{c^{2}} \frac{D_{LS}}{D_{S}}\sum_{j=1}^{n}M_j \frac{\vec{r}_{ij}}{r^2_{ij}} 
\end{equation}  
For the computation of the individual deflection angles the tree idea is used. This means that the positions of all lensing masses are sorted into a 2-dim. grid (in the lens plane). This grid is subdivided into 4 smaller squares until every cell contains only one mass. For the actual calculations of the deflection angle not every mass' gravitational influence is treated equally. Instead it is made use of the fact that the influence of lenses on the considered lightray falls off with $r^{-1}$ from the point where the ray hits the lens plane. Hence masses situated at higher distances, can be grouped together in the calculation of their gravitational potential, by multipole moments. This way the amount of computation time is reduced. 
When the deflection angles are determined, the deflected lightrays are traced to the source plane and a magnification map is created by counting the number of rays which hit each of its pixels. 
Once the map is created, the lightcurve can be obtained by specifying the transition path of the source within the map. The code thereby convolves the function describing the shape of the source with the magnification pattern of the map for each time step. Note that in the most observational cases in microlensing, it is the lens that is actually moving in front of a fixed background. 
For the model a it was desirable to work with a caustic geometry as simple as possible. Therefore only two point like masses where used as lenses.  
