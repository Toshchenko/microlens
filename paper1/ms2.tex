In the case of a high-quality lightcurve with insignificant noise and measurement errors, the parameters can be obtained
by simply identifying the characteristic instances of time without making use of the actual values of the magnification
 map. If the effect of the errors and the noise distorts the magnification time function enough so that the
characteristic epochs are not identifiable with the characteristic periods still visible, the boundaries of the periods
can be roughly estimated. Furthermore, if direct estimates of the parameters cannot be obtained we propose the use of a strong
statistical tool such as Markov-Chain Monte Carlo. In Section 6  we have studied the possibility of identifying a crescent
source and  the possibility of recovering the respective parameters. As magnification map, we have used a complex numerical
one generated with the microlensing code by \cite{1999A&A...346L...5W}. In addition, we have added to the signal a Gaussian
noise with an SNR of 1.6.  In the experiment, we have considered all nine combinations of original source profiles and assumed fitting source models.
Effectively we have fitted using MCMC all three sources with all three assumed fitting models. The results of the experiment allowed us to build a
procedure for distinguishing the shape of the source assuming that one of the three models we have considered is a good approximation.
The procedure would be useful for observers that endeavour to gain more information about an unresolved source for which they can study
the microlensing lightcurve.  As a first step in the procedure, one should first attempt to use MCMC with a Gaussian model assumption.
If either the value of the $\chi^2$ or the number of rejected datapoints is large then the next step is to change the assumed source model to
a uniform disc and redo the fitting. Finally, if the uniform disc assumption is rejected as well then the fitting should be done with a
crescent source model assumption. If at each of the three steps the data rejects the model then the source cannot be approximated by any of the three models.
Otherwise, if at one of the steps the data does not reject the model then the respective model is a good approximation.
