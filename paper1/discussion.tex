\section{Discussion}\label{sec:discussion}

In the current paper we simulate and study the resulting microlensing lightcurves of geometric crescent-shaped sources 
and compare them with the microlensing lightcurves of other simple mathematically describable source profiles. 
In order to mimic the behaviour of the flux of light from the source in the proximity of a fold caustic we make use of the simple approximation described in equation (5). 
The equation would exhaustively describe the magnification map and offer a good universal approximation for the 
particular microlensing regime that we consider. 
Namely, the shape of the caustic boundary in the proximity of the source in the respective plane can be approximated with
 a line due to reason that the local radius of curvature of caustic is orders of magnitude greater than the half-light
 radius of the studied source. 
In particular cases in which the previously mentioned approximation loses its validity the impact on the quality of the 
lightcurves is not evenly distributed. The shape of the lightcurve will be maintained. 
The data points corresponding to the source position before and during the overlapping of the caustic will be affected by
 smaller errors than the data points corresponding to later times. \\

The first two source profiles that we consider are the uniform disk and symmetric Gaussian source. 
Both of them can be described by a half-light radius $r_{1/2}$ and a total unlensed light flux $S_0$. 
With the two parameters constrained the one dimensional profiles as well as the lightcurves of the two source are completely determined, since no free parameter remains. 

The previous statement does not hold for a crescent source.
In the case of the crescent source there are in total five parameters: the integrated flux of the source $S_0$,
 the radii of the bright/dark disk $R_p$/$R_n$ and the displacement of the centers of the two disks on the axes perpendicular and parallel to the caustic $a$ and $b$. 

Two of the parameters can be reduced by expressing the results in terms of $S_0$ and $r_{1/2}$. 
The later being determinable for any set of parameters $R_p, R_n$ and $a^2+b^2$. Moreover, one of the displacement 
parameters $b$ has no impact on the one dimensional profile of the source that results from the projection of the source image on an axis perpendicular to the caustic. 

Since the one dimensional source profile that corresponds to an axis perpendicular to the caustic contains exhaustively all the information regarding the source that can be revealed by the lightcurve, 
the value of the parameter $b$ does not have an effect on the shape of the lighcurve. 

Nevertheless the $b$ parameter is relevant for the calculation of $r_{1/2}$. 
It's qualitative effect is to decrease the value of the half-light radius when the absolute value of the parameter is increased.  
With two parameters constrained and another irrelevant to the shape of the lightcurve, two free parameters remain $R_n$ and $a$. 
Figure 4 reveals that the lighcurve of a crescent source has more visible features than the other two light-curves 
corresponding to the disc and Guasssian shape. The parameters $R_n$ and $a$ have strong influences on the shape of 
the microlensed lightcurve, as can be seen in figures 5, 6 and 7. Moreover, the one dimensional source profile corresponding to the direction perpendicular to the caustic reveal four characteristic points. The overlap of each of these points with the caustic leaves visible features on the lightcurve at the corresponding instances of time. In timely order the instances correspond to the start of the overlap between the caustic and the bright disk, the start of the overlap between the caustic and the dark disk, the end of the overlap between the caustic and dark disk and finally the end of the overlap between the caustic and the bright disk.          


With the different source profiles and their corresponding lightcurves studied we can change our point of view of the system to that of an observer. 
The observer would basically detect only the lightcurve of such a source. As described in section 4.3 the timing of the onset and offset of the previously 
described periods can be used to estimate the values of the radii and one of the displacement parameters when assuming a geometric crescent. 
All quantities can be estimated in terms of the relative velocity of the source in a direction perpendicular to the caustic.


Furthermore, a simulated image of M87 presented in \citep{2012MNRAS.421.1517D} (figure 9) has been microlensed. On the resulting lightcurve the instances 
corresponding to the start and end of the black hole shadow and caustic overlap were distinguishable as can be seen in figure 10.
    

In case of a high quality lightcurve with insignificant noise and measurement errors, the parameters can be obtained 
by simply identifying the characteristic instances of time without making use of the actual values of the magnification
 map. If the effect of the errors and the noise distorts the magnification time function enough so that the 
characteristic epochs are not identifiable with the characteristic periods still visible, the boundaries of the periods 
can be roughly estimated. Furthermore, if direct estimates of the parameters cannot be obtained we propose the use of a strong
statistical tool such as Markov-Chain Monte Carlo. In Section 6  we have studied the possibility of identifying a crescent 
source and  the possibility of recovering the respective parameters. As magnification map we have used a complex numerical 
one generated with the microlensing code by \cite{1999A&A...346L...5W}. In addition we have added to the signal a Gaussian 
noise with a SNR of 1.6.  In the experiment we have considered all nine combinations of original source profiles and assumed fitting source models.
Effectively we have fitted using MCMC all three sources with all three assumed fitting models. The results of the experiment allowed us build a 
procedure for distinguishing the shape of the source assuming that one of the three models we have considered is a good approximation.  
The procedure would be useful for observers that endeavour to gain more information about an unresolved source for which they can study 
the microlensing lightcurve.  As a first step in the procedure, one should first attempt to use MCMC with a Gaussian model assumption. 
If either the value of the $\chi^2$ or the number of rejected datapoints is large then the next step is to change the assumed source model to 
an uniform disc and redo the fitting. Finally if the uniform disc assumption is rejected as well then the fitting should be done with a 
crescent source model assumption. If at each of the three steps the data rejects the model then the source cannot be approximated by any of the three models. 
Otherwise if at one of the steps the data does not reject the model then the respective model is a good approximation.   

The previously mentioned abstract parameters can be related to physical quantities specific to the central region of a quasar. 
As such the luminous region would correspond to the bright accretion disc that surrounds the black hole. The later's gravity would 
cause a shadow in the bright region limited by the extent of the event horizon of the black hole. \textbf{Therefore the radius 
of the bright disk would provide an estimate of the size of the accretion disk and the radius of the dark disc would provide
 and estimate of the gravitationally magnified Schwarzschild radius of the black hole $R_{S}^{magnified} = \Delta t_{dark} \cdot v_p$.
By $R_{S}^{magnified}$ we refer to the apparent Schwarzschild radius which is larger than the real value at large distances due to the
black hole's own gravity.  Moreover the gravitationally magnified value of the Schwarzschild radius is a monotonic function of the black 
hole's mass. Therefore it can be used to estimate the mass of the black hole if it was not rotating.} In the previous 
expression the $\Delta t_{dark}$ denotes the period of overlap between the black hole shadow and fold caustic. 
$v_p$ denotes the component of the relative velocity of the source and fold which is perpendicular to the caustic.  
The respective velocity is an unknown, though it can be constrained on a case by case basis to an order of magnitude 
or even better. This would require the study of the dynamics of the stellar structure which contains the gravitational lens. 
A better estimate of the relative velocity would facilitate a better estimate of the effective non-rotating black hole 
mass associated to the black hole shadow.  

    
The parameters whose values cannot be determined due to the loss of information from the directions parallel to the 
caustic could be obtained in the eventuality in which the same source crosses multiple caustics that are not parallel. 
Multiple crossing of caustics can reveal details of the one dimensional flux profile corresponding to multiple distinct 
directions which would allow the reconstruction of the two dimensional profile analogous to the process through which an image of a CT scan is obtained.  

 

