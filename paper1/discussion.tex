\section{Discussion}
[Mihai's contribution]\\
In the current paper we simulate and study the resulting microlensing lightcurves of geometric crescent-shaped sources 
and compare them with the microlensing lightcurves of other simple mathematically describable source profiles. 
In order to mimic the behaviour of flux of light from the source in the proximity of a fold caustic we make use of the simple approximation described in equation (5). 
The equation would exhaustively describe the magnification map and offer a good universal aproximation for the 
particular microlensing regime that we consider. 
Namely, the shape of the caustic boundary in the proximity of the source in the respective plane can be approximated with
 a line due to reason that the local radius of curvature of caustic is orders of magnitude greater than the half-light
 radius of the studied source. 
In particular cases in which the previously mentioned approximation loses its validity the impact on the quality of the 
lightcurves is not evenly distributed. The shape of the lightcurve will be maintained. 
The datapoints corresponding to the source position before and during the overlapping of the caustic will be affected by
 smaller errors than the datapoints corresponding to later times. \\

The first two source profiles that we consider are the uniform disk and symmetric gaussian source. 
Both of them can be described by a half-light radius $r_{1/2}$ and a total unlensed light flux $S_0$. 
With the two parameters constrained the one dimensional profiles as well as the lightcurves of the two source are completely determined, since no free parameter remains. 

The previous statement does not hold for a crescent source.
In the case of the crescent source there are in total five parameters: the integrated flux of the source $S_0$,
 the radii of the bright/dark disk $R_p$/$R_n$ and the displacement of the centers of the two disks on the axes perpendicular and parallel to the caustic $a$ and $b$. 

Two of the parameters can be reduced by expressing the results in terms of $S_0$ and $r_{1/2}$. 
The later being determinable for any set of parameters $R_p, R_n$ and $a^2+b^2$. Moreover, one of the displacemement 
parameters $b$ has no impact on the one dimensional profile of the source that results from the projection of the source image on an axis perpendicular to the caustic. 

Since the one dimensional source profile that corresponds to an axis perpendicular to the caustic contains exhaustively all the information regarding the source that can be revealed by the lightcurve, 
the value of the parameter $b$ does not effect the shape of the lighcurve. 

Nevertheless the $b$ parameter is relevant to the calculation of $r_{1/2}$. 
It's qualitative effect is to decrease the value of the half-light radius when the absolute value of the parameter is increased.  
With two parameters constrained and another irrelevant to the shape of the lightcurve, two free parameters remain $R_n$ and $a$. 
Figure 4 reveals that the lighcurve of a crescent source has more visible features than the other two light-curves 
corresponding to the disc and guasssian shape. The parameters $R_n$ and $a$ have strong influences on the shape of 
the microlensed lightcurve, as can be seen in figures 5, 6 and 7. Moreover, the one dimensional source profile corresponding to the direction perpendicular to the caustic reveal four characteristic points. The overlap of each of these points with the caustic leaves visible features on the lightcurve at the corresponding instances of time. In timely order the instances correspond to the start of the overlap between the caustic and the bright disk, the start of the overlap between the caustic and the dark disk, the end of the overlap between the caustic and dark disk and finally the end of the overlap between the caustic and the bright disk.          

With the different source profiles and their corresponding lightcurves studied we can change our point of view of the system to that of an observer. The observer would basically detect only the lightcurve of such as source. As described in section (Lightcurve of a crescent source) the timing of the onset and offset of the previously described periods can be used to estimate the values of the radii and one of the displacement parameters when assuming a geometric crescent, the halfwidths and one displacement parameters when assuming a constant brightness elliptical crescent and the halfwidth of the dark ellipse when assuming an elliptical gaussian crescent. All quantities can be estimated in terms of the relative velocity of the source in a direction perpendicular to the caustic.\\

The previously mentioned abstract parameters can be related to physical quantities specific to the central region of a quasar. Thus the luminous region would correspond to the bright accretion disc that surrounds the black hole. The later's gravity would cause a shadow in the bright region limited by the extent of the event horizon of the black hole. Therefore the radius of the bright disk would provide an estimate of the size of the accretion disk and the radius of the dark disc would provide and estimate of the Schwarzschild radius of the black hole. Moreover a value of the Schwarzschild radius 
can be used to estimate the mass of the black hole $M \approx \frac{\Delta t_{dark}  v_p c^2}{2G}$. In the previous 
expression the $\Delta t_{dark}$ denotes the period of overlap between the black hole shadow and fold caustic. 
$v_p$ denotes the component of the relative velocity of the source and fold which is perpendicular to the caustic.  Furthermore, a simulated image of M87 presented in \citep{2012MNRAS.421.1517D} (figure 9) has been microlensed. On the resulting lightcurve the instances corresponding to the start and end of the black hole shadow and caustic overlap were distinguishable as can be seen in figure 10.
    
The parameters whose values cannot be determined due to the loss of information from the directions parallel to the 
caustic could be obtained in the eventuality in which the same source crosses multiple caustics that are not parallel. 
Multiple crossing of caustics can reveal details of the one dimensional flux profile corresponding to multiple distinct 
directions which would allow the reconstruction of the two dimensional profile analogous to the process through which an image of a CT scan is obtained.  

In case of a high quality lightcurve with insignificant noise and measurement errors, the parameters can be obtained 
by simply identifying the characterstic instances of time withought making use of the actual values of the magnification
 map. If the effect of the errors and the noise distorts the magnification time function enough so that the 
characteristic points are not identifiable with the characteristic periods still visible, the boundaries of the periods 
can be roughly estimated. To study if good estimates of the parameters can still be obtained when datapoints are
 strongly affected by the noise we propose ....        



              

[Irshad's contribution]\\

This work presents a toy model to put useful constraints on the interior structure of quasars, particularly the environment geometry of the event horizon. To carry out this analysis on real data, one needs two situations: First, a quasar which can be observed with a reasonably good telescope with high signal to noise ($\sim 150$) and second, a caustic crossing event of the quasar. These events are not rare, however, to get an isolated caustic crossing might be. As the size of the source is negligible as compared to big caustic structures in the sky, even the single fold caustic crossing is not that rare.

In this toy analysis, we used very ideal data, which might not be the case with real source but we contaminated it to match a reasonable signal to noise. 

The main emphasis of this work was the crescent model parameters, which we show can be well recovered by the maximum likelihood analysis, apart from highly degenerate $a$ and $b$ parameters. The reason of this degeneracy is also very intuitive, if one integrates parallel to the caustic to get the integrated 1D brightness profile, so if the motion of the source is either parallel or perpendicular to the caustic, one of the two parameters ($a$ and $b$) is indistinguishable and in case of angled motion of the source, these two are highly degenerate. This degeneracy can be broken if multiple crossing events of the same quasar is available. So for example, one can use the first crossing to estimate the likelihood of the four parameters and use them as priors in the second crossing and so on. In this way, one can accurately and precisely recover all four parameters. Using our toy analysis we conclude that it is feasible, in practice, to recover 2D shape of the source using a microlensing event and distinguish between symmetric and asymmetric sources.



[Contributions from authors]\\
Prasenjit Saha provided the original idea and plan for the research project as well as multiple contributions to the analysis. Mihai Tomozeiu simulated and studied the ideal behaviour of the microlensing lightcurves for the different source profiles discussed. Joachim Wambsganss provided the numerical code used by Manuel Rabold to create the magnification map and corresponding lighcurves used in the MCMC analysis, analisis that was performed by Irshad Mohammed in the last part of the presented work. 
