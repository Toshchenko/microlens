\section{Discussion}

This work presents a toy model to put useful constraints on the interior structure of quasars. To carry out this analysis on real data, one needs two situations: First, a quasar which can be observed with a reasonably good telescope with high signal to noise and second, a caustic crossing event of the quasar. These events are not rare, however, to get an isolated caustic crossing might be. However, as the size of the source is negligible as compared to big caustic structures in the sky, even the single fold caustic crossing is not that rare.

In this toy analysis, we used very ideal data, which might not be the case with real source. So, we propose a more sophisticated analysis of this kind with different combinations of random and systematic noise and selection of data points for future.

The main emphasis of this work was the crescent model parameters, which we show can be well recovered by the maximum likelihood analysis, apart from highly degenerate $a$ and $b$ parameters. The reason of this degeneracy is also very intuitive, if one integrates parallel to the caustic to get the integrated 1D brightness profile, so if the motion of the source is either parallel or perpendicular to the caustic, one of the two parameters ($a$ and $b$) is indistinguishable and in case of angled motion of the source, these two are highly degenerate. This degeneracy can be broken if multiple crossing events of the same quasar is available. So for example, one can use the first crossing to estimate the likelihood of the four parameters and use them as priors in the second crossing and so on. In this way, one can accurately and precisely recover all four parameters.

Further, one can also study another interesting case, where the shape of the source is crescent like but the brightness of the crescent is not uniform but Gaussian.



